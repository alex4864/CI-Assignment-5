\documentclass[a4paper]{article}
\usepackage{amsmath, amssymb, bm}
\usepackage[margin=1in]{geometry}
\usepackage{graphicx}

\DeclareMathOperator*{\argmax}{arg\,max}

\begin{document}
	\begin{titlepage}
		\centering
		{\huge \bf Assignment 5\par}
		\vspace{1cm}
		{\Large Computational Intelligence, SS2018\par}
		\vspace{1cm}
		\begin{tabular}{|l|l|l|}
			\hline
			\multicolumn{3}{|c|}{\textbf{Team Members}}   \\ \hline
			Last name & First name & Matriculation Number \\ \hline
			Lee       & Eunseo     & 11739623             \\ \hline
			Shadley   & Alex       & 11739595             \\ \hline
			Lee       & Dayeong    & 11730321             \\ \hline
		\end{tabular}
	\end{titlepage}
	
	\section{Classification/ Clustering}
	\subsection{2 dimensional feature}
	\subsubsection{Perform all of the above-mentioned tasks for the EM algorithm.}
	\subsubsection{Perform all of the above-mentioned tasks for the K-means algorithm}
	\subsubsection{You may additionally choose any other pair of features; how would this change the classification accuracy}
	
	\subsection{4 dimensional feature}
	\subsubsection{How do the convergence properties and the accuracy of you classification change in comparison to scenario 2.1? }
	\subsubsection{Within your EM-function confine the structure of the covariance matrices to diagonal matrices! What is the influence on the result.}
	\subsection{Processing the data with PCA }
	\subsubsection{How much of the variance in the data is explained this way?}
	\subsubsection{How does the performance of your algorithms compare to scenario 2.1 and scenario 2.2?}
	\subsubsection{Apply PCA with whitening, so that the transformed data has zero mean and a unit covariance matrix. How does this influence the choice of your initialization?}
	
	\section{Samples from a Gaussian Mixture Model}
	\subsection{Write a function Y = sample-GMM(alpha, mu, cov, N)}
	\subsection{Using a GMM of your choice$ (K > 3)$, demonstrate the correctness of your function}
\end{document}